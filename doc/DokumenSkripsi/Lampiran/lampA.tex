%versi 3 (18-12-2016)
\chapter{Kode Program}
\label{lamp:A}

%terdapat 2 cara untuk memasukkan kode program
% 1. menggunakan perintah \lstinputlisting (kode program ditempatkan di folder yang sama dengan file ini)
% 2. menggunakan environment lstlisting (kode program dituliskan di dalam file ini)
% Perhatikan contoh yang diberikan!!
%
% untuk keduanya, ada parameter yang harus diisi:
% - language: bahasa dari kode program (pilihan: Java, C, C++, PHP, Matlab, C#, HTML, R, Python, SQL, dll)
% - caption: nama file dari kode program yang akan ditampilkan di dokumen akhir
%
% Perhatian: Abaikan warning tentang textasteriskcentered!!
%

\begin{comment}
\begin{lstlisting}[language=Java, caption=MyCode.c]

// This does not make algorithmic sense, 
// but it shows off significant programming characters.

#include<stdio.h>

void myFunction( int input, float* output ) {
	switch ( array[i] ) {
		case 1: // This is silly code
			if ( a >= 0 || b <= 3 && c != x )
				*output += 0.005 + 20050;
			char = 'g';
			b = 2^n + ~right_size - leftSize * MAX_SIZE;
			c = (--aaa + &daa) / (bbb++ - ccc % 2 );
			strcpy(a,"hello $@?"); 
	}
	count = ~mask | 0x00FF00AA;
}

// Fonts for Displaying Program Code in LATEX
// Adrian P. Robson, nepsweb.co.uk
// 8 October 2012
// http://nepsweb.co.uk/docs/progfonts.pdf

\end{lstlisting}


\lstinputlisting[language=Java, caption=MyCode.java]{./Lampiran/MyCode.java} 
\end{comment}

\lstinputlisting[language=Java, caption=Dijkstra.java, label={code:dijkstra}]{./Lampiran/Dijkstra.java}

\lstinputlisting[language=Java, caption=AStar.java, label={code:astar}]{./Lampiran/AStar.java}

\lstinputlisting[language=Java, caption=FloydWarshall.java, label={code:floydwarshall}]{./Lampiran/FloydWarshall.java}

\lstinputlisting[language=Java, caption=ShorestPathStrategy.java, label={code:sps}]{./Lampiran/ShorestPathStrategy.java}

\lstinputlisting[language=Java, caption=Worker.java, label={code:worker}]{./Lampiran/Worker.java}

\lstinputlisting[language=Java, caption=ServiceListener.java, label={code:sl}]{./Lampiran/ServiceListener.java}

\lstinputlisting[language=PHP, caption=main.php, label={code:mainphp}]{./Lampiran/main.php}

\lstdefinelanguage{JavaScript}{
  keywords={break, case, catch, continue, debugger, default, delete, do, else,
    finally, for, function, if, in, instanceof, new, return, switch, this,
    throw, try, typeof, var, let, const, while, with, yield, await, async},
  keywordstyle=\color{blue}\bfseries,
  ndkeywords={class, export, boolean, throw, implements, import, this},
  ndkeywordstyle=\color{darkgray}\bfseries,
  identifierstyle=\color{black},
  sensitive=false,
  comment=[l]//,
  morecomment=[s]{/*}{*/},
  commentstyle=\color{gray}\ttfamily,
  stringstyle=\color{red}\ttfamily,
  morestring=[b]',
  morestring=[b]",
  morestring=[b]`,
}

\lstinputlisting[language=JavaScript, caption=main.js, label={code:mainjs}]{./Lampiran/main.js}

\lstinputlisting[language=JavaScript, caption=protocol.js, label={code:protocol}]{./Lampiran/protocol.js}

\lstinputlisting[language=PHP, caption=Api.php, label={code:api}]{./Lampiran/Api.php}