\chapter{Kesimpulan dan Saran}
\label{chap:kesimpulandansaran}
Bab ini akan membahas kesimpulan dari hasil pengembangan perangkat lunak KIRI. Selain itu, akan disampaikan juga saran untuk pengambangan lanjutan yang dapat dilakukan.
\section{Kesimpulan}
Berdasarkan hasil penelitian dan implementasi serta pengujian yang telah dilakukan dalam pengembangan perangkat lunak KIRI, maka dapat disimpulkan hal-hal berikut.
\begin{enumerate}
    \item Pada Tugas Akhir ini, perubahan arsitektur kelas pada NewMenjangan untuk menerapkan \textit{strategy pattern} berhasil dilakukan. Penerapan dilakukan dengan membentuk sebuah \textit{superclass} bernama \texttt{ShortestPathStrategy} sebagai antarmuka strategi. Kelas ini berfungsi sebagai abstraksi algoritma pencarian jalur, sehingga memungkinkan berbagai algoritma dapat diimplementasikan secara fleksibel sebagai turunan dari kelas-kelas algoritma pencarian jalur.
    \item Implementasi algoritma A-Star dan Floyd-Warshall sebagai \textit{concrete strategy} telah berhasil dilakukan. Pada NewMenjangan dibuat dua kelas java baru, yaitu \texttt{AStar.java} dan \texttt{FloydWarshall.java} untuk mengimplementasikan kedua algoritma tersebut. Kedua kelas tersebut juga menjadi kelas turunan dari kelas \texttt{ShortestPathStrategy} dan dapat dipertukarkan sesuai kebutuhan sistem. Selain itu, dari hasil pengujian menunjukan bahwa kedua algoritma tersebut dapat berjalan dengan baik dan dapat menghasilkan jalur yang optimal setelah menerima masukan-masukan yang dibutuhkan.
\end{enumerate}

\section{Saran}
Berdasarkan hasil penelitian dan implementasi serta pengujian yang telah dilakukan dalam pengembangan perangkat lunak KIRI, berikut ini adalah beberapa saran yang dapat disampaikan.
\begin{enumerate}
    \item Menerapkan algoritma \textit{shortest path} lainnya, selain Dijkstra, A-Star, dan Floyd-Warshall, guna memperluas cakupan analisis performa dan potensi peningkatan efisiensi dalam pencarian jalur terpendek, seperti algoritma Bellman-Ford, algoritma Johnson's, dan lain-lain.
    \item Melakukan eksplorasi untuk pendekatan berbasis geometri sebagai alternatif pengganti representasi \textit{node}, dengan tujuan untuk mengurangi jumlah \textit{node} yang terdapat pada graf-graf jalur sehingga jumlah node yang diproses ketika melakukan pencarian jalur terpendek menjadi lebih sedikit dan  dapat meningkatkan performa.
\end{enumerate}